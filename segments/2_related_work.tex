\begin{frame}{Tấn công DNNs - FGM, I-FGM}
    \begin{itemize}
        \item Kí hiệu $\mathbf{x_0}$ và $\mathbf{x}$ lần lượt là mẫu gốc và mẫu đối nghịch,
        $t$ là lớp mục tiêu cần tấn công.

        \item Tấn công dựa trên $L_{\infty}$
        \begin{equation}
            \mathbf{x} = \mathbf{x_0} - \epsilon \times \text{sign}(\nabla J(\mathbf{x_0}, t))
            \label{eq:1}
        \end{equation}
        với $\epsilon$ là độ biến dạng $L_{\infty}$ giữa $\mathbf{x}$ và $\mathbf{x_0}$ và 
        $\text{sign}(\nabla J)$ là dấu của gradient
        
        \item Tấn công dựa trên $L_1$ và $L_2$
        \begin{equation}
            \mathbf{x} = \mathbf{x_0} - \epsilon \frac{\nabla J(\mathbf{x_0}, t)}
            {\lVert \nabla J(\mathbf{x_0}, t) \rVert _q}
            \label{eq:2}
        \end{equation}
        với $q = 1,2$ và $\epsilon$ là độ méo tương quan
    \end{itemize}
\end{frame}

\begin{frame}{Tấn công DNNs - C\&W}
    \begin{itemize}
        \item Thay vì sử dụng hàm mất mát trên tập huấn luyện Carlini và Wagner
        đã thiết kế  một hiệu chỉnh $L_2$ trong hàm mất mát dựa trên lớp logit trong DNNs để sinh 
        ra các mẫu đối nghịch (Carlini and Wagner 2017b)
        \item Công thức này hóa ra là một trường hợp riêng của thuật toát EAD (sẽ được trình bày trong phần sau)
    \end{itemize}
\end{frame}

\begin{frame}{Phòng thủ DNNs}
    \begin{itemize}
        \item Defensive distillation - Chưng cất phòng thủ (Papernot et al.2016b)
        \item Adversarial training - Huấn luyện đối nghịch (Zheng et al. 2016; Madry et al. 2017; 
        Tram`er et al. 2017; Zantedeschi, Nicolae, and Rawat 2017)
        \item Detection methods - Phương pháp dò tìm (Feinman et al. 2017; Grosse et al. 2017; Lu, Issaranon, and Forsyth 2017; 
        Xu, Evans, and Qi 2017)
    \end{itemize}
\end{frame}